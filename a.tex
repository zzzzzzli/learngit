\documentclass[UTF8]{ctexart}
\usepackage[left=1cm,right=1cm,top=1cm,bottom=1cm]{geometry}
\begin{document}
你好,现在开始学习 Git 分布式版本控制系统。\par
\begin{enumerate}
  \item
  \begin{itemize}
    \item \verb|git init|, 初始化目录为一个 Git 仓库
    \item \verb|git add|, 添加文件到 Git, 准备提交到仓库
    \item \verb|git commit -m "wirte somthing about your changes"|, 提交文件到仓库
  \end{itemize}
  \item 修改你的文件
  \begin{itemize}
    \item \verb|git status| 告诉你哪个文件被修改过
    \item \verb|git diff| 告诉你文件哪里被修改成什么样子了
    \item \verb|git add, git commit -m "xxx"| 把修改过的文件扔到仓库里
  \end{itemize}
  \item 版本退回
  \begin{itemize}
    \item \verb|git log| 或者 \verb|git log --pretty=oneline| 查看历史版本的 ID 号,以及你的备注(就是commit -m "")后面写的东西
    \item \verb|git reset HEAD^^| 退回到上两个版本,后面几个尖角号就是退回到前面第几个版本
    \item \verb|git reflog| 现在退回到旧版本了,但如果又想重返未来,而log文件里面刚才最新的版本已经消失了,此时,要么在命令窗口里往上翻,找到那个版本的 ID, 如果你把窗口关了,重新打开之后,也能在reflog文件里查看未来的版本记录
    \item \verb|git reset --hard ID| 退回到 ID 指定的版本,ID 是你要退回的版本号,在log或者reflog里面
  \end{itemize}
  \item 专有名词
  \begin{description}
    \item[工作区] 就是文件存放的目录,手动修改文件的地方
    \item[版本库] 工作区同目录下的 .git 文件夹,里面存放了分支、指向分支的指针 HEAD, 以及暂存区
    \item[暂存区] 每次修改文件之后,先把修改过的文件加入到暂存区之后,再一起提交到分支
  \end{description}
  \item 撤销修改
  \begin{itemize}
    \item \verb|git checkout -- file| 撤销工作区的修改,注意,两个短线左右都有空格,或者暂存区还未提交的文件被修改了,也可以撤销修改
    \item \verb|git reset HEAD file| 如果修改过的文件被放到暂存区了,那么用这个命令再拿回来,然后再用上一条命令撤销修改
  \end{itemize}
\end{enumerate}
\end{document}
